\documentclass[10pt,letter]{article}
\usepackage{amsmath,amssymb,breqn,enumitem,fullpage,graphicx,setspace,mathtools,pst-node,stmaryrd,tikz-cd}
\onehalfspacing
\usepackage{fullpage}

\begin{document}
\begin{center} 
{\bf Petersen Ch. 3 Exercises} \\
Holly Mandel $\bullet$ 5/25/20
\end{center}
\noindent{\bf 7.  Parallel Ricci tensor implies constant scalar curvature} $\bullet$ If $\nabla \text{Ric} = 0$, then for any $p \in M$ and $X$, $Y$, $Z \in T_p M$, 
\[
(\nabla_Z R(E_i,\cdot,\cdot, E_i))(X,Y) = \sum_i Z(R(E_i,X,Y,E_i))-R(E_i,\nabla_Z X,Y,E_i)-R(E_i, X,\nabla_Z Y,E_i) = 0,
\]
where $(E_i)$ is an orthonormal frame in a neighborhood of $p$. Thus by the symmetries of $R$ (Prop. 3.1.1),
\begin{dmath*}
Z\bigg( \sum_{i,j}  R(E_i,E_j,E_j,E_i)\bigg) = \sum_{i,j} 2 R(E_i,\nabla_Z E_j,E_j,E_i).
\end{dmath*}
But we can assume $(E_i)$ is a parallel frame by Exercise 2.5.19. Therefore the righthand term vanishes, so the scalar curvature is constant.\\
 
\noindent{\bf 13. A manifold with enough totally geodesic hypersurfaces has constant curvature} $\bullet$ Let $N$ be any vector in $T_pM$ for some $p \in M$, and let $X$, $Y$ and $Z$ be tangent vectors to totally geodesic hypersurface $H$ which has $N$ as a normal vector at $p$. By Theorem 3.2.5 and the vanishing of $\Pi$, $g(R(X,Y)Z,N) = -g(R(X,Y)N,Z) = 0$. This implies that $R(X,Y)N = 0$. But since $N$ can be any vector and $X$, $Y$ any two vectors orthogonal to it, the result follows from Exercise 3.4.10. \\

\noindent{\bf 18. Scaling the metric} $\bullet$ It is easily seen that the connection which is metric with respect to $g$ is also metric with respect to $\lambda^2 g$. Therefore the Levi-Civita connection for these two metrics is the same. It then follows immediately from the definition that the $(3,1)$ tensor is unchanged. The rest of the relations are verified by direct calculation, being careful to incorporate the scaling of orthonormal bases as well as the scaling of the metric.\\

\noindent{\bf 28. Cartan formalism} $\bullet$ (1.) Because $\nabla$ is metric an $(E_i)$ is an orthonormal frame, for any $i,j,k$, 
\begin{dmath*}
0 = E_k(g(E_i,E_j)) = g(\nabla_k E_i,E_j) + g(E_i,\nabla_k E_j)
= \omega^j_i(E_k) + \omega^i_j(E_k),
\end{dmath*} 
so $\omega^j_i = -\omega^i_j$. 

Now \begin{dmath*}
(d\omega^i)(E_p,E_q) = \omega^i([E_p,E_q]) 
= \omega^i(\nabla_p E_q - \nabla_q E_p)
= \omega_p^i(E_q) - \omega_q^i(E_p).
\end{dmath*}
On the other hand,
\begin{dmath*}
\omega^j \wedge \omega^i_j(E_p,E_q) = \delta_{jp} \omega^i_j(E_q) - \delta_{jq} \omega^i_j(E_q) 
= \omega^i_p(E_q) - \omega^j_q(E_p).
\end{dmath*}
Therefore $d\omega^i = \omega^j \wedge \omega^{i}_j$. 

(2.) We compute
\begin{dmath*}
\Omega^j_i(E_p,E_q)E_j = \nabla_p \nabla_q E_i - \nabla_q \nabla_p E_i - \nabla_{[p,q]} E_i
= \nabla_p(\omega_i^k(E_q) E_k) - \nabla_q(\omega_i^k(E_p) E_k) - \omega_i^k([E_p,E_q]) E_k
= \omega_i^k(E_q) \omega_k^\ell(E_p) E_\ell + E_p(\omega_i^k(E_q)) E_k - 
\omega_i^k(E_p) \omega_k^\ell(E_q) E_\ell - E_q(\omega_i^k(E_p)) E_k- \omega_i^k([E_p,E_q]) E_k
= d\omega_i^k(E_p,E_q)E_k +(\omega_i^k(E_q) \omega_k^\ell(E_p)+\omega_i^k(E_p) \omega_k^\ell(E_q)) E_\ell.
\end{dmath*}
Applying $\omega^j$ to both sides yields
\begin{dmath*} \Omega^j_i(E_p,E_q) =d\omega_i^j(E_p,E_q) + (\omega_i^k(E_q) \omega_k^j(E_p)+\omega_i^k(E_p) \omega_k^j(E_q)) = d\omega_i^j(E_p,E_q) - (\omega_i^k \wedge \omega_k^j)(E_p,E_q),
\end{dmath*}
which is the desired formula.

(3.) Computations omitted. 
\end{document}