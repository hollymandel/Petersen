\documentclass[10pt,letter]{article}
\usepackage{amsmath,amssymb,breqn,enumitem,fullpage,graphicx,setspace,mathtools,pst-node,stmaryrd,tikz-cd}
\onehalfspacing
\usepackage{fullpage}

\begin{document}
\begin{center} 
{\bf Petersen Ch. 3 Exercises} \\
Holly Mandel $\bullet$ 5/25/20
\end{center}
\noindent{\bf 7.  Parallel Ricci tensor implies constant scalar curvature.} $\bullet$ If $\nabla \text{Ric} = 0$, then for any $p \in M$ and $X$, $Y$, $Z \in T_p M$, 
\[
(\nabla_Z R(E_i,\cdot,\cdot, E_i))(X,Y) = \sum_i Z(R(E_i,X,Y,E_i))-R(E_i,\nabla_Z X,Y,E_i)-R(E_i, X,\nabla_Z Y,E_i) = 0,
\]
where $(E_i)$ is an orthonormal basis for $T_pM$. Thus by Prop. 3.1.1,
\begin{dmath*}
Z\bigg( \sum_{i,j}  R(E_i,E_j,E_i,E_j)\bigg) = \sum_{i,j} 2 R(E_i,\nabla_Z E_j,E_j,E_i).
\end{dmath*}
But we can assume $(E_i)$ is a parallel frame by Exercise 2.5.19. Therefore the righthand term vanishes, so the scalar curvature is constant. 
 
\end{document}